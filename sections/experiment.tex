This section presents a comprehensive evaluation of the xyz (system's name) system through quantitative performance metrics, user studies, and challenging real-world scenarios. Our evaluation methodology combines technical benchmarking with human-robot interaction assessment to validate both system capabilities and user experience.

\subsection{Performance Evaluation Protocol}

We evaluate our system across multiple dimensions to ensure robust validation:

\begin{itemize}
\item \textbf{Success rate analysis}: Measured across 20? pick-and-place trials
\item \textbf{Temporal performance}: End-to-end latency from natural language input to task completion
\item \textbf{Component timing breakdown}: Individual analysis of LLM processing on Jetson AGX Orin, vision detection, and motion execution phases
\item \textbf{Complex linguistic understanding}: Evaluation using challenging prompts such as \textit{"Pick up the metallic tool specifically designed for driving or tightening screws"}
\item \textbf{Repeatability assessment}: Consistency testing through 5 repeated trials across 5 different objects (25 total evaluations)
\end{itemize}

\subsection{Spatial Reasoning and Multi-Object Scenarios}

To assess the system's spatial understanding and multi-object manipulation capabilities:

\begin{itemize}
\item \textbf{Single object manipulation}: Baseline performance with isolated objects
\item \textbf{Multi-object selection}: Disambiguation between multiple visible objects (\textit{"pick the red cube, not the blue one"})
\item \textbf{Complex spatial relationships}: Advanced spatial reasoning tasks (\textit{"place the red cube to the left of the blue one"})
\item \textbf{Overlapping object challenges}: Performance degradation analysis in cluttered environments
\item \textbf{Relative positioning tasks}: Evaluation of spatial preposition understanding (above, below, beside, between)
\end{itemize}

\subsection{Human Subject Evaluation}

We evaluate system usability through a user study with 12? participants (4 novice, 4 intermediate, 4 expert users):

\begin{itemize}
\item \textbf{Performance metrics}: Task success rate and completion time
\item \textbf{Usability survey}: System Usability Scale (SUS) questionnaire (need to design the survey if we want to include this)
\item \textbf{User feedback}: Post-task interviews on system transparency and trust (need to design interview questions if we want to include this)
\end{itemize}

\subsection{Statistical Analysis}

We ensure reliable results through:

\begin{itemize}
\item \textbf{Adequate sample sizes}: Sufficient trials for meaningful statistical analysis
\item \textbf{Error reporting}: Confidence intervals for all performance metrics
\item \textbf{Significance testing}: Standard statistical tests (t-tests, ANOVA) where appropriate
\end{itemize}

\subsection{Distributed System Performance}

We evaluate the distributed architecture's effectiveness across hardware components:

\begin{itemize}
\item \textbf{Network latency analysis}: ROS2 topic communication delays on Jetson AGX Orin
\item \textbf{Local inference performance}: LLM processing times on dedicated Jetson hardware  
\item \textbf{Offline operation validation}: System functionality without external network dependencies
\item \textbf{Multi-device coordination}: Synchronization effectiveness across distributed nodes
\item \textbf{Resource utilization}: CPU, GPU, and memory usage on Jetson during inference tasks
\end{itemize}